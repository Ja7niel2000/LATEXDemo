\documentclass{beamer}
\definecolor{colorA}{RGB}{216,217,20}
\definecolor{colorB}{RGB}{12,45,72}
\definecolor{colorC}{RGB}{29,29,166}
\setbeamercolor*{palette primary}{bg=colorC}
\setbeamercolor*{palette secondary}{bg=colorB, fg = white}
\setbeamercolor*{palette tertiary}{bg=colorA, fg = white}
\setbeamercolor*{titlelike}{fg=colorA}
\setbeamercolor*{background canvas}{bg=colorB}
\setbeamercolor{normal text}{fg=white}  % todo el texto azul

\usepackage[numbers]{natbib}
\usepackage{graphicx} % Required for inserting images



\title{\Huge Tarea semanal 2\\[-1cm]}
\author{Jatniel Carranza Bolaños\\[-0.9cm]}
\date{Agosto 2025}
\titlegraphic{\includegraphics[height=2cm]{img/Trifuerza.svg.png}}

\begin{document}
    % Parte 1, Portada
    \begin{frame}
        \titlepage
        \vspace{0.5cm}
        \begin{center}
            {\small \textit{\cite{zeldaImg}}(La TriFuerza, elemento representativo de la saga The Legend of Zelda de Nintendo)}
        \end{center}
    \end{frame}
    
    %Parte 2, Teorema
    \begin{frame}{Teorema}
        \begin{center}
            \text{\cite{analisisMatematico}}{$\forall a \in R, -a=-1(a)   $}
        \end{center}
          Este teorema pertenece al área de álgebra.\\ 
          Trata sobre las propiedades de los números reales y cómo se comportan las operaciones básicas, en este caso la multiplicación y la adición.\\ 
          Lo elegí porque fue el primer teorema que pude demostrar.
    \end{frame}
    
    %Parte 3, Ecuación
    \begin{frame}{Resolver la ecuación $\tfrac{3}{4}x+9=15$}
        \begin{align*}
            \tfrac{3}{4}x+9 &= 15 &\textit{Ecuación Original} \\
            4(\tfrac{3}{4}x+9) &= 15 \cdot 4 &\textit{Multiplicación por 4}\\
            (4 \cdot \tfrac{3}{4}x) + (4 \cdot 9) &= 60 &\textit{Distributiva}\\
            3x+36&=60&\textit{Simplificación}\\
            3x+36-36 &=60 -36 & \textit{Restando 36}\\
            3x &=24 & \textit{Simplificación}\\
            \tfrac{1}{3}\cdot3x &=\tfrac{1}{3}\cdot24 & \textit{Multiplicando por $\tfrac{1}{3}$}\\
            x &=8& \text{Resultado final}
        \end{align*}
    \end{frame}
    % BIBLIOGRAFÍA
    \begin{frame}[allowframebreaks]
        \frametitle{Bibliografía}
        \bibliographystyle{plainnat}
        \bibliography{referencias}
    \end{frame}


    
\end{document}
